\documentclass{exam} 
\usepackage[latin1]{inputenc}
\usepackage[cyr]{aeguill}
\usepackage[francais]{babel}

\usepackage{exercise}

\renewcommand\listexercisename{Liste des exercices}%
\renewcommand\ExerciseName{Exercice}%
\renewcommand\AnswerName{Solution de l'exercice}%
\renewcommand\ExerciseListName{Ex.}%
\renewcommand\AnswerListName{Solution}%
\renewcommand\ExePartName{Partie}%
\renewcommand{\ExerciseHeader}{%
  \par\noindent
  \hskip -7mm
  \textbf{ \ExerciseName\, \ExerciseHeaderNB \ExerciseHeaderTitle \ExerciseHeaderOrigin}%
  \par\nopagebreak\medskip
}

\usepackage{float}
\usepackage{graphicx}

\graphicspath{{./imgs/}}

\usepackage{xcolor} % colour pour inkscape
\usepackage{caption} % caption sans figure envionnement


\usepackage{amsfonts,amssymb,amsmath,amsthm} 
\newtheorem*{remark}{Remarque}
\newtheorem*{definition}{D�finition}
\newtheorem*{lemma}{Lemme}

\usepackage[hidelinks]{hyperref} 
\usepackage[a4paper,margin=1in]{geometry}

\pagestyle{head}
\lhead{Pr�paration Concours EPF}
\chead{}
\rhead{2017-18}


\title{}
\author{}
\date{}

\def\vphi{\varphi}
\def\dd{\mathrm{d}}

\usepackage{exercise}
\usepackage{bm}

\begin{document} 

\thispagestyle{head}
%\vspace*{10pt}



\paragraph{Question 13.}
\begin{center}
{\it La d�riv�e de la fonction $f(x) = (2x+1)^{1/3}$ ?}
\end{center}

On a : $f=g\circ h$ o� $g(x) = x^{1/3}$ et $h(x) = 2x+1$.
Puis, 
\[
f'(x) = h'(x)\,g'(h(x))= 2\, g'(2x+1) = 2\, \frac{1}{3} (2x+1)^{1/3-1} = {\color{blue}  \frac{2}{3}(2x+1)^{-2/3}}.
\]



\paragraph{Question 17.}
\begin{center}
$\int_2^3 \frac{1}{(1+x)^3}\, \dd x$ ?
\end{center}

On a 

\[
\int_2^3 \frac{1}{(1+x)^3}\, \dd x = \left[ -\frac{1}{2}(1+x)^{-2} \right]_2^3 = -\frac{1}{2}\left(\frac{1}{16}-\frac{1}{9}\right) = {\color{blue}\frac{7}{288}}. 
\]


\paragraph{ Question 60.}
\begin{center}
{\it Une primitive de $f(x)=\sin x \cos x$ est $g(x)=\frac{\sin^2x}{2}$ ?}
\end{center}

On a:
\[
g'(x) = 2\frac{1}{2}\sin x \cos x = \sin x \cos x  = f(x).
\]

{\color{blue} Vrai}.

\end{document}