\documentclass{exam} 
\usepackage[latin1]{inputenc}
\usepackage[cyr]{aeguill}
\usepackage[francais]{babel}

\usepackage{exercise}

\renewcommand\listexercisename{Liste des exercices}%
\renewcommand\ExerciseName{Exercice}%
\renewcommand\AnswerName{Solution de l'exercice}%
\renewcommand\ExerciseListName{Ex.}%
\renewcommand\AnswerListName{Solution}%
\renewcommand\ExePartName{Partie}%
\renewcommand{\ExerciseHeader}{%
  \par\noindent
  \hskip -7mm
  \textbf{ \ExerciseName\, \ExerciseHeaderNB \ExerciseHeaderTitle \ExerciseHeaderOrigin}%
  \par\nopagebreak\medskip
}

\usepackage{float}
\usepackage{graphicx}

\graphicspath{{./imgs/}}

\usepackage{xcolor} % colour pour inkscape
\usepackage{caption} % caption sans figure envionnement


\usepackage{amsfonts,amssymb,amsmath,amsthm} 
\newtheorem*{remark}{Remarque}
\newtheorem*{definition}{D�finition}
\newtheorem*{lemma}{Lemme}

\usepackage[hidelinks]{hyperref} 
\usepackage[a4paper,margin=1in]{geometry}

\pagestyle{head}
\lhead{Pr�paration Concours EPF}
\chead{}
\rhead{2017-18}


\title{}
\author{}
\date{}

\def\vphi{\varphi}
\def\dd{\mathrm{d}}

\usepackage{exercise}
\usepackage{bm}

\begin{document} 

\thispagestyle{head}
%\vspace*{10pt}



\paragraph{Question 18.}
\begin{center}
{\it Conjugu� de $z= (3+i)/(1-i)$ ? }
\end{center}

On a 
\[
z =\frac{3+i}{1-i} = \frac{3+i}{1-i} \frac{1+i}{1+i} = \frac{3+3i+i-1}{1-i^2} = \frac{2+4i}{2} = 1+2i
\]

Le conjugu� d'un nombre complexe  sous forme alg�brique $a+ib$ est $a-ib$, donc 
\[
{\color{blue} \bar z = 1-2i}.
\]


\paragraph{Question 32.}
\begin{center}
{\it -6 = ?}
\end{center}

On a : $e^{i\pi} = -1$ donc $-6 = -1 \times 6 = e^{i\pi} \times  6 = {\color{blue} 6\, e^{i\pi}}$.\\ 

\paragraph{ Question 40.}
\begin{center}
{\it Sur $[0,2\pi]$, $\sin x$ est du signe de ?}
\end{center}

On a
$\sin x$ est positif sur $[0,\pi]$ et n�gatif sur $[\pi,2\pi]$.
\begin{enumerate}
\item Non car $1+\cos x\ge 0, \; \text{ pour tout } x\in [0,2\pi]$. 
\item Non car $1-\cos^2x = \sin^2x \ge 0 \; \text{ pour tout } x\in [0,2\pi]$.
\item Non car $\cos(\pi/2)-1=-1<0$ qui est du signe oppos� � $\sin(\pi/2)=1>0$.
\item Oui car $\sin x + \sin^2 x = \sin x\, (1+\sin x)$ et $1+\sin x\ge 0 \; \text{ pour tout } x\in [0,2\pi]$
donc $\sin x$ est su signe de ${\color{blue} \sin x + \sin^2 x}$.\\
\end{enumerate}

\paragraph{Question 61.}
\begin{center}
{\it Exprimer $\arg z$ en fonction de $\arg \bar z$ ?}
\end{center}

On a : $\arg \bar z = -\arg z  \; [2\pi]$. Donc si $\theta=\arg z\; [2\pi]$ alors
\[
\arg \bar z = {\color{blue} -\theta \; [2\pi]}. 
\]



\end{document}